\author{Dave Footitt}
\title{Characters}
\date{02/04/2018}
\documentclass[12pt,a4paper]{article}
\usepackage{palatino,url}
\pagenumbering{gobble}

%
% This doc just serves as background info for the players, simply a paragraph or two each.  Actual skills etc are listed on proper character sheets.
%

\begin{document}
\begin{center}
\section*{Characters}
\line(1,0){200}
\end{center}
\vspace{1em}
\subsection*{Frank Morris}
The death of old friend Bumpy has prompted you to get back in touch with your two fellow escapees Clarence and John and finally find out just what's going on.\\\\
Something was definitely amiss there; inhuman noises in the night, prisoners just disappearing without trace and you're sure one or two guards were in on it.  Bumpy gave you a lot of help escaping back in '62 and his death has rekindled your interest in the whole affair.\\\\
Sure you're a bit of a bruiser, but you ain't no dummy either.  You're going to get to the bottom of it alright, and trusting nobody other than your fellow escapees.
%Morris reportedly ranked in the top 2\% of the general population in intelligence, as measured by IQ testing, displaying an IQ of 133.\\
%Frank Lee Morris (September 1, 1926 disappeared June 11, 1962) was born in Washington, D.C.. He was orphaned at age 11, and spent most of his formative years in foster homes. He was convicted of his first crime at age 13, and by his late teens had been arrested for crimes ranging from narcotics possession to armed robbery.[9][10] Morris reportedly ranked in the top 2% of the general population in intelligence, as measured by IQ testing, displaying an IQ of 133. He served time in Florida and Georgia, then escaped from the Louisiana State Penitentiary while serving 10 years for bank robbery. He was recaptured a year later while committing a burglary, and sent to Alcatraz in 1960 as inmate number AZ1441.
\subsubsection*{Inventory}
\begin{itemize}
\item{Torch}
\item{S\&W model 10 .38 revolver (6 shots in barrel, plus 12 bullets)}\\\textbf{1d10} damage
\end{itemize}
\newpage
\subsection*{John Anglin}
The death of old friend Bumpy shocked you, and when fellow escapee Frank Morris got in touch with you and your brother you decided to return to Alcatraz to have a sniff around.\\\\
Something was definitely amiss there; inhuman noises in the night, prisoners just disappearing without trace and you're sure one or two guards were in on it.  Bumpy gave you a lot of help escaping back then and his death has rekindled your interest in the whole affair.\\\\
You and your brother Clarence are very close and were virtually inseparable as youngsters.  You are both skilled swimmers, sure came in handy in '62!\\\\
Quick in, quick out, is your plan and if anything or anyone looks suspect, you and Clarence come first.  Freedom was a long time coming and feels good!
%Unlike Frank, you and your brother are not hardened criminals - you come from a big family, times were tough and you needed money.  Most of the places you robbed were closed in order to avoid injury.
\subsubsection*{Inventory}
\begin{itemize}
\item{Torch}
\item{Baseball bat; \textbf{1d6} damage}
\end{itemize}

%The Anglin brothers, John William (May 2, 1930 disappeared June 11, 1962) and Clarence (May 11, 1931 disappeared June 11, 1962) were born into a family of thirteen children in Donalsonville, Georgia. Their parents, George Robert Anglin and Rachael Van Miller Anglin, were seasonal farm workers; in the early 1940s, they moved the family to Ruskin, Florida, 20 miles south of Tampa, where the truck farms and tomato fields provided a more reliable source of income. Each June they would migrate north as far as Michigan to pick cherries. Clarence and John were reportedly inseparable as youngsters; they became skilled swimmers, and amazed their siblings by swimming in the frigid waters of Lake Michigan as ice still floated on its surface.

%They began robbing banks and other establishments as a team in the early 1950s, usually targets that were closed, to ensure that no one got injured. They claimed that they used a weapon only once, during a bank heist, a toy gun.[11] They were arrested in 1956. Both received 15 to 20 year sentences, which they served at Florida State Prison, Leavenworth Federal Penitentiary, and then Atlanta Penitentiary.

%After repeated failed attempts to escape from the Atlanta facility, the brothers were transferred to Alcatraz.[12] John arrived on October 21, 1960, as inmate AZ1476, and Clarence on January 10, 1961, as inmate AZ1485.
\newpage
\subsection*{Clarence Anglin}
The death of old friend Bumpy shocked you, and when fellow escapee Frank Morris got in touch with you and your brother you decided to return to Alcatraz to have a sniff around.\\\\
Something was definitely amiss there; inhuman noises in the night, prisoners just disappearing without trace and you're sure one or two guards were in on it.  Bumpy gave you a lot of help escaping back then and his death has rekindled your interest in the whole affair.\\\\
You and your brother John are very close and were virtually inseparable as youngsters.  You are both skilled swimmers, sure came in handy in '62!\\\\
Quick in, quick out, is your plan and if anything or anyone looks suspect, you and John come first.  Freedom was a long time coming and feels good!
%Unlike Frank, you and your brother are not hardened criminals - you come from a big family, times were tough and you needed money.  Most of the places you robbed were closed in order to avoid injury.
\subsubsection*{Inventory}
\begin{itemize}
\item{Crowbar}
\item{Switchblade; \textbf{1d4} damage}
\end{itemize}
\newpage
\subsection*{Doctor}
That newspaper article you read certainly piqued your interest - you've had your suspicions about that place for a while, and the rumours aren't going away.  Perhaps something really \textit{has} been going off there!\\\\
A naturally inquisitive sort, you've been intrigued by the place since reading about the prison break in '62, and you often wonder if they really did get away with it.\\\\
Besides, what harm could it really do just to get over there and have a look around?
\subsubsection*{Inventory}
\begin{itemize}
\item{First aid kit (3 uses; restores \textbf{1d4} health)}
\item{Walking stick (with retractable blade); \textbf{1d6} damage}
\end{itemize}
\newpage
\subsection*{Detective} % Detective working for someone or other, hired to investigate what's going on.  Normal detective skills.
Working for Pinkertons, you've been sent to go and sniff around.  S'what you're good at, right?\\\\
Load of old twaddle anyway in your opinion; quick in, poke about, and out again.  Piece of cake.  Should have the ol' report written up in no time.\\\\
An old hand like you, however, knows you can't be too careful, so you've brought along your trusty Colt 45 just in case.
\subsection*{Inventory}
\begin{itemize}
\item{Torch}
\item{Colt .45 (6 shots in barrel, plus 12 bullets); \textbf{1d10+2} damage}
\end{itemize}
\newpage
\subsection*{Inventor}
Aha! \textit{this} looks more like it!  Mystery goings on, and a nice deserted area will give you a good chance to test out your new bit of kit!\\\\
Nothing you like more than a bit of adventure, and if there really is anything going on in there it'll raise your profile a bit!\\\\
You'd better get yerself over there with your new flamethrower.  Buckle up, and may God himself help anyone down there, heheh
\subsection*{Inventory}
\begin{itemize}
\item{Infra-red goggles}
\item{Home-made Flamethrower; \textbf{2d6+burn} damage; malfunction on 92+}
\end{itemize}
%%%%%% RESERVE CHARACTERS %%%%%%%%
%\newpage
%\line(1,0){200}
%\subsection*{Captured Journalists (Reserves; both similar backgrounds)}
%Kept alive for experiments.  Introduce if someone dies.  Doesn't require weapon or anything.
\end{document}
